\begin{abstract}[flattitle]
基于深度学习的医学图像的语义分割取得了极大的进展,但模型的训练通常依赖于大量的像素级别的标注,所需的标注成本很高。为了降低标注成本和难度,弱监督语义分割是一个很有前景的方向。本文探索两种形式的弱监督学习,基于弱标签和基于噪声标签的语义分割。

在基于弱标签的语义分割中,先前的方法虽然探索了二维图像上的弱标签,却很少利用三维体积图像的结构先验引导。为了解决这个问题,我们提出了一种新的弱监督分割策略,能够在模型预测和学习中更好地捕捉三维形状的先验。我们的主要思想是,通过利用弱标签来提取一个自学习的形状表示,然后将这个形状表示纳入分割预测中进行形状改进。为此,我们设计了一个深度卷积网络,由一个语义分割模块和一个形状去噪模块组成,并通过迭代学习策略进行训练。更进一步,我们为体积图像引入了一种混合式标签设计的弱标注方案,在不增加整体标注成本的情况下,提高了模型的学习效果。在三个不同形状的器官分割基准的实验表明,我们的方法性能优于最近的先进方法。值得注意的是,我们甚至可以仅用10\%的弱标记切片实现强大的性能,相比其他方法有明显的优势。

在基于噪声标签的语义分割中,多数现有的方法在分割中忽略了像素间相关性和结构性先验,通常在物体边缘区域产生噪声预测。为此,我们采用了基于超像素的数据表示,并设计了一种鲁棒的迭代学习策略,将有噪声感知的分割网络训练和噪声标签改进相结合,两者都由超像素引导。这种设计使我们能够利用分割标签的结构约束,有效地降低模型学习中标签噪声的影响。在两个数据集的实验表明,我们的方法在多种噪声水平下都超越了现有工作,达到了最高的分割性能,并且有很强的鲁棒性和泛化性能。

\textbf{关键词:} 语义分割,弱监督学习,形状先验,自学习方法,基于噪声标签的学习

\end{abstract}

\begin{abstract*}[flattitle]
With deep learning methods, semantic segmentation of medical images have been making great progress. However, these methods usually requires a large amount of training data with high-quality pixelwise annotations. To reduce the label difficulty and cost, weakly supervised segmentation is a promising approach. In this thesis, we explore two kinds of weakly supervised semantic segmentation: based on weak labels or noisy labels.

Prior methods based on weak labels, while often focusing on weak labels of 2D images, exploit few structural cues of volumetric medical images. To address this, we propose a novel weakly-supervised segmentation strategy capable of better capturing 3D shape prior in both model prediction and learning. Our main idea is to extract a self-taught shape representation by leveraging weak labels, and then integrate this representation into segmentation prediction for shape refinement. To this end, we design a deep network consisting of a segmentation module and a shape denoising module, which are trained by an iterative learning strategy. Moreover, we introduce a weak annotation scheme with a hybrid label design for volumetric images, which improves model learning without increasing the overall annotation cost. The empirical experiments show that our approach outperforms existing SOTA strategies on three organ segmentation benchmarks with distinctive shape properties. Notably, we can achieve strong performance with even 10\% labeled slices, which is significantly superior to other methods. 

In the area of learning segmentation from noisy labels, Most existing methods neglect the pixel correlation and structural prior in segmentation, often producing noisy predictions around object boundaries. To address this, we adopt a superpixel representation and develop a robust iterative learning strategy that combines noise-aware training of segmentation network and noisy label refinement, both guided by the superpixels. This design enables us to exploit the structural constraints in segmentation labels and effectively mitigate the impact of label noise in learning. Experiments on two benchmarks show that our method outperforms recent state-of-the-art approaches, and achieves superior robustness in a wide range of label noises.

\textbf{Keywords: } Semantic Segmentation, Weakly Supervised Learning, Shape Prior, Self-taught Learning, Learning with Noisy Labels

\end{abstract*}