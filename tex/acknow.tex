\ifgraduate
\begin{resume}
李帅霖,山西运城人,上海科技大学 2018 级硕士研究生。
\end{resume}

\begin{education}
2018.9 - 至今:      \quad 硕士 \quad 上海科技大学 \quad 计算机科学与技术

2014.9 - 2018.6     \quad 学士 \quad 南方科技大学 \quad 计算机科学与技术
\end{education}

\begin{publications}
%   论文发表…… (非匿名环境)
\begin{enumerate}
    \item Li, Shuailin, Zhitong Gao, and Xuming He. "Superpixel-Guided Iterative Learning from Noisy Labels for Medical Image Segmentation." International Conference on Medical Image Computing and Computer-Assisted Intervention (MICCAI). Springer, Cham, 2021.
    \item He, Qian, Shuailin Li, and Xuming He. "Weakly Supervised Volumetric Segmentation via Self-taught Shape Denoising Model." Medical Imaging with Deep Learning (MIDL). 2021.
    \item Li, Shuailin, Chuyu Zhang, and Xuming He. "Shape-aware semi-supervised 3d semantic segmentation for medical images." International Conference on Medical Image Computing and Computer-Assisted Intervention (MICCAI). Springer, Cham, 2020.
\end{enumerate}

\end{publications}

\begin{publications*}
%   论文发表…… (匿名环境)
\end{publications*}

% \begin{patterns}
%   专利申请或授权记录…… (非匿名环境)
% \end{patterns}

% \begin{patterns*}
%   专利申请或授权记录…… (匿名环境)
% \end{patterns*}

% \begin{projects}
%   个人参与的科研项目、获奖情况…… (仅非匿名环境显示)
% \end{projects}
\fi


\begin{acknowledgement}
岁月不居,时节如流,研究生生涯踏入尾声。
这三年多的时间里,有初入研究的懵懂,有踌躇不前的苦闷,有好高骛远的落差,也有不达不休的决心,有柳暗花明的欣喜,有再接再厉的勇气。
人在环境中成长,我很感谢自己所处的这个环境。导师何旭明教授给予了专业而深入的科研训练,使我从糊里糊涂的新人成长为略有心得的研究生。实验室的同学们,使我在良好的研究氛围中不断借鉴学习。上科大的丰富资源,提供了全面而重要的支持。
我向这三年多来遇到的每一个人致谢,他们或是博学亲切的教授,或是乐于助人的同学,或是默默付出的宿管,或是关护温和的前辈。
最后,感谢亲人们的关心、支持与鼓励,感谢与我探讨过诸多问题的朋友们,他们都使我受益匪浅,乐观应对生活这一课题。
回望身后,所历颇多,但曙光初露,这个故事,才刚刚开始呢。

\end{acknowledgement}